\documentclass{article}
\author{AIMS, Bayesian Statistics 2009}
\title{Some common distributions}

\usepackage{amsmath,amssymb}

\usepackage{Sweave}
\begin{document}

\maketitle

\section{Continuous distributions}


\subsection{Uniform}


\begin{itemize}
\item[] Notation $x \sim U(a,b)$

\item[] Parameters: Boundaries $a$, $b$, with $b > a$ ($-\infty < a < \infty$, $-\infty < b < \infty$

\item[] Density: $f(x) - \frac{1}{b-a}, \mbox{ for } x \in [a,b]$

\item[] Summary: $E[x]=0$, $Var(x)=\frac{(b-1)^2}{12}$, no mode
\end{itemize}


\subsection{Normal}

\begin{itemize}
\item[] Notation: $x \sim N(\mu, \sigma^2)$

\item[] Parameters: Location $\mu$ ($-\infty < \mu < \infty$), scale $\sigma > 0$

\item[] Density $p(x) = \frac{1}{\sqrt{2 \pi} \sigma} \exp \left( -\frac{1}{2 \sigma^2}(x-\mu)^2 \right) \mbox{ for } -\infty < x < \infty$

\item[] Summary: $E[x]=\mu$, $Var(x)=\sigma^2$, $mode(x)=\mu$
\end{itemize}





\subsection{Chi-square ($\chi^2$)}

If $X \sim N(0,1)$, then $X^2 \sim \chi^2_1$.   Also, if $X_1, X_2, \ldots, X_p$ are all iid $\sim N(0,1)$ then $\sum_{j=1}^p X_j^2 \sim \chi^2_p$.

\begin{itemize}
\item[] Notation: $x \sim \chi^2_\nu$

\item[] Parameters: Degrees of freedom $\nu>0$

\item[] Density: $f(x) = \frac{2^{\nu/2}}{\Gamma(\nu/2)} x^{\nu/2-1}e^{-x/2} \mbox{ for } 0 < x < \infty$

\item[] Summary: $E[x]=\nu$, $Var[x]=2\nu$, $mode(x)=\nu-2$ (for $\nu>2$)
\end{itemize}

(The $\chi^2$ is a special form of the Gamma distribution - what parameters of the Gamma distribution give you a $\chi^2$?)

\subsection{F distribution}

\begin{itemize}

\item[] Notation $x \sim F_{\nu_1,\nu_2}$

\item[] Parameters: Degrees of freedom $\nu_1$ and $\nu_2$ ($\nu_1>0$, $\nu_2>0$)

\item[] Density: $f(x)=\frac{\nu_1^{\nu_1/2}\nu_2^{\nu_2/2}}{B(\frac{1}{2}\nu_1,\frac{1}{2}\nu_2)} \frac{x^{\nu_1/2-1}}{(\nu_2+\nu_1 x)^{(\nu_1+\nu_2)/2}}  \mbox{ for } 0 < x < \infty$

\item[] Summary: $E[x]=\nu_2/(\nu_2-2)$, $Var(x)=\frac{2 \nu_2^2(\nu_1+\nu_2-2)}{\nu_1(\nu_2-2)^2(\nu_2-4)}$, $mode(x)=\frac{\nu_2}{\nu_2+1} \frac{\nu_1-2}{\nu_1}$

\end{itemize}

Note if $W_1 \sim \chi^2_{\nu_1}$ and  $W_2 \sim \chi^2_{\nu_2}$ then $X=\frac{W_1/\nu_1}{W_2/\nu_2} \sim F_{\nu_1, \nu_2}$




\subsection{Exponential}

\begin{itemize}
\item[] Notation: $x \sim Expon(\theta)$

\item[] Parameters: Inverse scale $\theta$ where $\theta>0$

\item[] Density: $f(x)=\theta e^{-\theta x} \mbox{ for } 0 < x < \infty$

\item[] Summary: $E[x]=\frac{1}{\theta}$, $Var(x)=\frac{1}{\theta^2}$, $mode(x)=0$
\end{itemize}

Notes: This is equivalent to a Gamma with $a=?$ and $r=\theta$
Notes: The density can also be written in the form $\frac{1}{\lambda}e^{-\frac{1}{\lambda} x}$.



\subsection{Gamma}

\begin{itemize}
\item[] Notation $x \sim Gamma(a,r)$

\item[] Parameters: Shape $a>0$, rate $r>0$ (the $rate = 1/scale$ and might be called inverse scale)

\item[] Density: $f(x)=\frac{r^{a}}{\Gamma(a)}x^{(a-1)}e^{-xr}$ for $x>0$.

\item[] Summary $E[x]=\frac{a}{r}$, $Var(x)=\frac{a}{r^2}$, $mode=\frac{a-1}{r}$ for $a \geq 1$
\end{itemize}

You can also see the Gamma parameterised slightly differently (with a scale rather than a rate)

\begin{itemize}

\item[] Parameters: Shape $a>0$, scale $s>0$

\item[] Density $f(x)=\frac{(\frac{1}{s})^{a}}{\Gamma(a)}x^{(a-1)}e^{-\frac{x}{s}}$  for $x>0$

\end{itemize}



\subsection{Beta}

\begin{itemize}
\item[] Notation: $x \sim Beta(a,b)$

\item[] Parameters: ``Prior sample size'' $a>0$, $b>0$

\item[] Density: $f(x) = \frac{\Gamma(a+b)}{\Gamma(a) \Gamma(b)} x^{a-1} (1-x)^{b-1} \mbox{ for } 0 \leq x \leq 1$

\item[] Summary: $E[x]=\frac{a}{a+b}$, $Var(x)=\frac{ab}{(a+b)^2(a+b+1)}$, $mode(x)=\frac{a-1}{a+b-2}$
\end{itemize}

\subsection{Student t}

\begin{itemize}

\item[] Notation: $x \sim t_\nu(\mu, \sigma^2)$.   $t_\nu$ is an abbreviation for $t_\nu(0,1)$

\item[] Parameters: Degrees of freedom $\nu > 0$, location $\mu$ ($-\infty < \mu < \infty$), scale $\sigma > 0$.

\item[] Density: $f(x)=\frac{\Gamma((\nu+1)/2)}{\Gamma(\nu/2)\sqrt{\nu \pi \sigma}} \left( 1 + \frac{1}{\nu} \left( \frac{x-\mu}{\sigma} \right)^2\right)^{-(\nu+1)/2} \mbox{ for } -\infty < x < \infty $

\item[] Summary: $E[x]=\mu$ for $\nu>1$, $Var(x)=\frac{\nu}{\nu-2}\sigma^2$ for $\nu>2$, $mode(x)=\mu$

\end{itemize}


\subsection{Log-Normal}

\begin{itemize}

\item[] Notation (not common) $x \sim LN(\mu, \sigma)$.   Apparently this is sometimes called the Galton distribution.

\item[] Parameters $\mu$ (the mean of the variable's natural logarithm), $\sigma^2$ (the variance of the variable's natural logarithm).

\item[] Density function $f(x|\mu, \sigma) = \frac{1}{x \sigma \sqrt{2 \pi}} e^{-\frac{(\log(x) - \mu)^2}{2 \sigma^2}}$

\item[] Summary: $E[Y] = \exp(\mu + \frac{\sigma^2}{2})$, $Var(Y) = (\exp(\sigma^2-1)\exp(2\mu + \sigma^2)$

\end{itemize}

\section{Discrete distributions}


\subsection{Binomial}

\begin{itemize}
\item[] Notation: $x \sim Bin(n, \pi)$

\item[] Parameters: $\pi$, probability of ``success'' ($\pi \in [0,1]$).   We also need the integer $n$.

\item[] Density: $f(x) = \binom{n}{x} \pi^x(1-\pi)^{n-x}; \mbox{ for } x=0,1,\ldots,n$

\item[] Summary: $E[x]=n\pi$, $Var(x)=n\pi(1-\pi)$, $mode(x)=floor\left\{(n+1)p\right\}$
\end{itemize}


\subsection{Poisson}

\begin{itemize}
\item[] Notation: $x \sim Poisson(\lambda)$

\item[] Parameters: rate=$\lambda>0$

\item[] Density: $f(x)=\frac{\lambda^xe^{-\lambda}}{x!},\ x=0,1,2,\ldots$

\item[] Summary: $E[x]=\lambda$, $Var(x)=\lambda$, $mode=floor(\lambda)$
\end{itemize}


\subsection{Negative Binomial}

\begin{itemize}

\item[] Notation: $x \sim NB(a,b)$

\item[] Parameters: Shape $a>0$, inverse scale $b>0$

\item[] Density: $f(x)= \binom{x+a-1}{a-1}  \left( \frac{a}{a+1} \right)^a \left( \frac{1}{b+1} \right)^x$ for $x=0, 1, 2, \ldots$

\item[] Summary: $E[x]=\frac{a}{b}$, $Var(x)=\frac{a}{b}(b+1)$

\end{itemize}

\subsection{Beta-binomial}

\begin{itemize}

\item[] Notation: $x \sim Beta-Bin(n, a, b)$

\item[] Parameters: ``Prior sample size'' $a>0$, $b>0$.   We also need to know the integer $n$

\item[] Density: $f(x)=\frac{\Gamma(n+1)}{\Gamma(x+1)\Gamma(n-x+1)} \frac{\Gamma(a+x) \Gamma(n+b-x)}{\Gamma(a+b+n)} \frac{Gamma(a+b)}{\Gamma(a)\Gamma(b)}$ for $x=0,1,2, \ldots, n$

\item[] Summary: $E[x]=n\frac{a}{a+b}$, $Var(x)=n\frac{ab(a+b+n)}{(a+b)^2(a+b+1)}$

\end{itemize}

\end{document}
