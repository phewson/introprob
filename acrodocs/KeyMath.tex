\documentclass{article}
\usepackage{amsmath,amssymb}
\title{Some important ``math'' facts}
\author{Paul Hewson}

\usepackage{Sweave}
\begin{document}
\sffamily
\maketitle

\section{Binomial Theorem}

\begin{displaymath}
\sum_{x+0}^n \binom{n}{x} p^x q^{n-x} = (p+q)^n 
\end{displaymath}

\section{Geometric Series}

\begin{displaymath}
\sum_{x=0}^{\infty} a^x = \frac{1}{1-q}
\end{displaymath}

(for $|q| < 1$)

\section{Euler's Number}

These are standard expressions

\subsection{Taylor Series version}

\begin{displaymath}
\sum_{x=0}^{\infty} \frac{a^x}{x!} = e^{a}
\end{displaymath}

\subsection{``Compound Interest'' version}

\begin{displaymath}
\left( 1 + \frac{a}{n} \right)^n = e^a
\end{displaymath}

\section{Integration by parts}

(a very brief reminder)

\begin{displaymath}
\int u\ dv = uv - \int v\ du
\end{displaymath}

\section{The Gamma Function}

\begin{displaymath}
\Gamma (a) = \int_0^{\infty} x^{a-1} e^{-x} dx
\end{displaymath}
for any real $a>0$

\subsection{Nice properties}

\begin{itemize}
\item[] $\Gamma (a) = (a-1)!$ (for positive integers
\item[] $\Gamma(a+1) = a \Gamma(a)$ (recursive)
\item[] $\Gamma(\frac{1}{2}) = \sqrt{\pi}$
\end{itemize}

\end{document}

