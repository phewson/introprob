\documentclass{book}
\title{Probability with Applications: Course Notes}
\author{Paul Hewson and Malgorzata Wojtys}
\date{February - May 2016}
\usepackage{amsmath,amssymb,amsfonts,graphics,hyperref,color,setspace,synttree,keystroke}

\usepackage{longtable}
\usepackage{a4wide,color,Sweave,url,verbatim,xargs,longtable}
\usepackage{tikz}
\newenvironment{question}{\item \textbf{Problem}\newline}{}
\newenvironment{solution}{\comment}{\endcomment}
\newenvironment{answerlist}{\renewcommand{\labelenumi}{(\alph{enumi})}\begin{enumerate}}{\end{enumerate}}


\begin{document}
\sffamily
\onehalfspacing

\maketitle

\chapter{Background to the course}


\section{Why study probability?}

There is a perception (often gained by taking A level modules) that probability is statistics.   Please don't confuse the two.   Statistics is a process of drawing inference about a population from a sample of data, and yes, it uses many techniques from various branches of mathematics including probability.   But probability is a mathematical field in its own right, the logic of uncertainty.   There are various reasons for studying probability:

\begin{itemize}
\item In order to gain a mature mathematical understanding of statistical inference
\item In order to understand physics (and particularly quantum physics)
\item In order to develop various computer science algorithms, such as the so-called Monte Carlo methods
\item In order to understand gambling and finance (I believe these are separate topics, but there are lucrative careers to be made as an actuary)
\item Applied probability in its own right
\end{itemize}

This module is intended as a formal introduction to the branch of mathematics known as probabilty.   We will illustrate applied probability where possible to keep the module alive.   It is \emph{not} a course in statistics.   Discrete probability allows you to use some very simple arithmetic to solve some very baffling problems and is a good way of developing rigorous mathematical reasoning skills.   Probability generally touches on several branches of mathematics and this course will give you the opportunity to practice your skills in areas such as calculus.

\section{Background Definitions}

There is a fascinating philosophical / linguistic / epistemological history behind the concept of probability.   For our purposes, \emph{Probability} is nothing more than a number between $0$ and $1$ that behaves in accordance with three axioms.   However, you can do a lot with numbers.   You can think of probability as being:

\begin{itemize}
\item A symmetrical thing (the coin has two faces, heads and tails)
\item A long run frequency thing (out of 10,000 coin tosses, 5,023 were heads)
\item A subjective thing (that person is obvious a crook and if they want me to bet on tails it is very likely they have a rigged coin)
\end{itemize}




\chapter{Brief review of Sets}






\subsubsection{Set theory notation}

% I might not cover this in a lecture, we need the first six rows at the moment, the rest are only needed if we were going to do a lot of proofs and some very formal probability theory.
 
% For completeness, we quote a table from Grimmett and Stirzaker (2004) which gives a listing of set theory and probability theory notation.\\[0.5in]

% {\color{blue}
% \begin{footnotesize}
% \begin{tabular}{lll}
% Notation & Set theory & Probability Theory \\
% \hline
% $\Omega$ & A collection of objects & Sample space \\
% $\omega$ & A subset of $\Omega$ & Elementary event/outcome\\
% $A$ & A subset of $\Omega$ & Event that some condition $A$ occurs \\
% $A^C$ & Complement of $A$ & No event $A$ occurs \\
% $A \cap B$ & Intersection of $A$ and $B$ & Both $A$ and $B$ occur \\
% $A \cup B$ & Union of $A$ and $B$ & Either $A$ occurs, or $B$ occurs or both
% \\
% $A \setminus B$ & Difference & $A$ occurs but $B$ does not occur \\
% $ A \bigtriangleup B $ & Symmetric difference & Either A or B but not both \\
% $A \subseteq B$ & Inclusion & If $A$ occurs then $B$ occurs \\
% $\emptyset$ & Empty set & Impossible event \\
% $\Omega$ & Whole space & Certain event
% \end{tabular}
% \end{footnotesize}
% }




\section{Sample spaces: how we use Sets}

\textit{More information is given in Section 1.2 of Blitzstein and Hwang}

Here are some key definitions:

\fbox{
  \begin{minipage}[c]{1.0\linewidth}
An \emph{experiment} %(esperimento) 
is any process that generates a set of outcomes where the outcome is uncertain.   For a random experiment:

\begin{itemize}
\item The \emph{sample space} $\Omega$ is the set of all possible outcomes 
%(campione di spazio)
\item An \emph{event} is a subset of the sample space 
%(evento)
  \begin{itemize}
  \item A \emph{simple event} is an event which cannot be a union of other events
  \item A \emph{composite event} is an event which is not a simple event
  \end{itemize}
\item The \emph{event space} is the set of all events (evento di spazio)
\end{itemize}
\end{minipage}
}

\subsection{Example}

Consider an experiment in which three coins are tossed.   We are interested in the {\color{red}\emph{event}} that all coins will show the same face. 



The following tree diagram denotes the breakdown of the sample space ($\Omega$).   After the first coin toss we have either a head or a tail.   After the second, for either of the possible outcomes of the first toss we also have a head or tail.   Finally, after the third coin toss, we have a row denoting that a head or tail is possible for any of the previous outcomes. 

%% Cardano was the first person to start thinking about systematically evaluating sample spaces in this way, but he missed a trick in that to him HTH and HHT would be the same (two heads and a tail).   It took nearly 100 years and Galileo to really master the business of sample spaces.  


\subsection{Tree diagram}

\begin{figure}[!h]
\synttree[$\Omega$ [H [H [H][T]][T [H][T]]]  [T [H [H][T]][T [H][T]]]] 
\end{figure}


In total therefore, the tree diagram shows on the bottom row that there are a total of $2^3$ possible outcomes in the sample space.   In other words, there are 8 different ways of travelling though the tree diagram,  resulting in the following outcomes: 
\begin{itemize}
\item {\color{red}HHH}, HHT, HTH, HTT, THH, THT, TTH, {\color{red}TTT}.
\end{itemize}  



 We are interested in the event that all faces are the same.   There are two outcomes in the sample space which correspond to this event - denoted in red.\\[0.25in]   

\fbox{
\begin{minipage}[c]{3.5in}
Recap:
\begin{itemize}
\item {\color{blue}Sample Space} ($\Omega$) - the eight outcomes for the experiment: 
\\{\color{red}HHH}, HHT, HTH, HTT, THH, THT, TTH, {\color{red}TTT}. 
\item {\color{blue}Event} ($A$) - the two outcomes of interest: {\color{red}HHH}, {\color{red}TTT}
\end{itemize}
\end{minipage}
}
\vspace{0.25in}

There are 8 outcomes in our sample space, two of them correspond to the event ``all three coins show the same space''.   If we continue to focus on naive definitions of probability, it pays us to consider Venn diagrams.


% We can form collections $\mathfrak{B}$ of (overlapping or non-overlapping) subsets of our sample space $\Omega$.   There are some restrictions on these collections which must be:

% \begin{itemize}
% \item The empty set $\emptyset$ must be in the collection $\mathfrak{B}$
% \item If $A \in \mathfrak{B}$ then $A^C \in \mathfrak{B}$
% \item If $A_1, A_2, \ldots \in \mathfrak{B}$ then $\cup_{i=1}^\infty A_i \in \mathfrak{B}$
% \end{itemize}

% These might seem trivially obvious for the naive problems, and hopefully we won't need to say any more.   For example, the second condition tells you if you have a world cup final of Argentina versus Germany, you can't say the complement of Argentina winning is Brasil winning however happy Brasilian fans might be with a German victory.


\subsection{Venn diagram}

An alternative method of visualising is via a Venn Diagram.

\begin{center}
\begin{figure}[!h]
\begin{picture}(200,110)
\put(0,0){\framebox(200,110){}}
\qbezier(15,55)(15,105)(65,105)
\qbezier(65,105)(115,105)(115,55)
\qbezier(115,55)(115,5)(65,5)
\qbezier(65,5)(15,5)(15,55)

%\qbezier(80,55)(80,105)(130,105)
%\qbezier(130,105)(180,105)(180,55)
%\qbezier(180,55)(180,5)(130,5)
%\qbezier(130,5)(80,5)(80,55)
\put(5,95){$\Omega$}
\put(40,50){$A$ (HHH, TTT)}
\put(130,50){$A^C$ }
\put(120,30){ (HTH HTH HHT }
\put(120,10){ THT TTH THH)}
%\put(80,50){ $A \cap B$}
\end{picture}
\end{figure}
\end{center}


In this diagram, $A$ denotes the event ``all three faces show the same'', and $A^C$ denotes ``A complement'' (also known as ``not A''), in other words the six outcomes where the faces are not identical.  




\subsection{Multiple events}

For the same experiment, consider again the event $A$, ``all three coins show the same face''.   But now, consider a second event, $B$, which we define as ``two or more coins show a head''

\begin{itemize}
\item A: {\color{red}HHH}, HHT, HTH, HTT, THH, THT, TTH, {\color{red}TTT}.
\item B: {\color{blue}HHH}, {\color{blue}HHT}, {\color{blue}HTH}, HTT, {\color{blue}THH}, THT, TTH, TTT
\end{itemize}

As before, 2 out the 8 outcomes correspond to event $A$.   We can also see that 4 of the 8 outcomes correspond to event $B$.   We can also see that one outcome (HHH) corresponds to both event $A$ and event $B$.   We can extend the Venn diagram to illustrate this:


\begin{center}
\begin{figure}[!h]
\begin{picture}(200,110)
\put(0,0){\framebox(200,110){}}
\qbezier(15,55)(15,105)(65,105)
\qbezier(65,105)(115,105)(115,55)
\qbezier(115,55)(115,5)(65,5)
\qbezier(65,5)(15,5)(15,55)

\qbezier(80,55)(80,105)(130,105)
\qbezier(130,105)(180,105)(180,55)
\qbezier(180,55)(180,5)(130,5)
\qbezier(130,5)(80,5)(80,55)
\put(5,95){$\Omega$}
\put(40,50){A}
\put(35,35){(TTT)}
\put(130,50){B}
\put(125,35){(THH HTH}
\put(125,20){ HHT)}
\put(80,50){ $A \cap B$}
\put(80,35){ (HHH)}
\end{picture}
\end{figure}
\end{center}



You can see that we use the Venn diagram to visually illustrate that the outcome HHH is in both event A and event B.   Standard set theory notation $A \cap B$ is used to denote that this is an ``intersection''.

\begin{itemize}
\item Which outcomes are in $(A \cup B)^C$?
\end{itemize}
%% don't know what to do here, these are the three events that are in neither A nor B





We should be able to verify standard operations such as:
\begin{itemize}
\item $A \cap B$: HHH
\item $A \cup B$: HHH, TTT, HTH, HHT, THH
\item $B^C$: TTT, TTH, THT, HTT
\item $(A \cup B)^C$: TTH, THT, HTT
\end{itemize}



John Venn has in interesting family background, see \href{http://www-groups.dcs.st-and.ac.uk/history/Biographies/Venn.html}{\color{blue}MacTutor article on John Venn}.

%% I don't think I'd cover this in a lecture

\subsubsection{Commutation and association}

Some mathematical properties of sets.


Commutativity:
\begin{itemize}
\item $A \cup B = B \cup A$
\item $A \cap B  = B \cap A$
\end{itemize}


Associativity:
\begin{itemize}
\item $A \cup B \cup C = (A \cup B) \cup C = A \cup (B \cup C)$
\item $A \cap B \cap C = (A \cap B) \cap C = A \cap (B \cap C)$
\end{itemize}

Distributivity:
\begin{itemize}
\item $A \cap (B \cup C) = (A \cap B) \cup (A \cap C)$
\item $A \cup (B \cap C) = (A \cup B) \cap (A \cup C)$
\end{itemize}


And De Morgan's laws say:
\begin{itemize}
\item $(A \cup B)^C = A^C \cap B^C$
\item $(A \cap B)^C = A^C \cup B^C$
\end{itemize}



This material will get covered more formally in a pure maths course, hopefully we can see how these can be used as the basis for probability.



\section{Problems}
\begin{enumerate}
\input{Problems/dummy1/Set1_BasicSets.tex}
\input{Problems/dummy1/Set1_SampleSpaces.tex}
\input{Problems/dummy1/Set1_CoinTossing.tex}
\end{enumerate}



\chapter{The naive definition of probability}

\section{Na\'ive definition of Probability}

\textit{This is covered in section 1.3 of Blitzstein and Hwang}

Consider a random experiment with a finite number of outcomes, each with the same probability.   For event $A$, we claim that:

\begin{displaymath}
\tiny
\mbox{Probability of an event} = \frac{\mbox{Number of possible sample points consistent with this event}}{\mbox{Total number of sample points}}
\end{displaymath}

Or in a better notation:

\begin{equation}
p[A] = \frac{n[A]}{n[\Omega]}
\end{equation}
where $A$ denotes the event we are interested in, $n[A]$ is the number of ways in which $A$ can happen and $n[\Omega]$ is the sample space.  


Recall from our first Venn diagram, where we used $A^C$ to denote the \emph{complement} of $A$ (i.e. those outcomes that weren't in $A$), then the probability of $A^C$ (probability that $A$ doesn't happen), $p[A^C]$ is:

\begin{displaymath}
p[A^C] = 1 - p[A]
\end{displaymath}


Likewise, if we consider the second Venn diagram, and consider that $p[B]$ denotes the probability of event $B$ happening we have:

\begin{itemize}
\item $P[A \cup B]$: probability that event $A$ or event $B$, or both happens
\item $P[A \cap B]$: probability that both $A$ and $B$ happen
\end{itemize}


\chapter{The Axioms of Probability}

The system of probability we have been working through took a long time to develop, and wasn't fully formalised until the 1930's by Kolmogorov (see 
\href{http://www.gap-system.org/~history/Biographies/Kolmogorov.html}{\color{blue}www.gap-system.org/~history/Biographies/Kolmogorov.html}).   He stated three fundamental Axioms, from which all other results can be derived.


\section{Axioms of probability}

\fbox{
  \begin{minipage}[c]{1.0\linewidth}
\begin{itemize}
\item $p[A] \geq 0$ for any event $A$
\item $p[\Omega] = 1$ where $\Omega$ is the sample space 
\item If $[A_i];i=1,2,\ldots$ are mutually exclusive then $p[A_1] \cup p[A_2] \cup \ldots = p[A_1] + p[A_2] + \ldots$
\end{itemize}
\end{minipage}
}

Mutually exclusive means that $A_i \cap A_j = \emptyset$ for all $i \neq j$


We shall examine the implications of these Axioms in terms of obtaining mathematical functions that can serve as models for probability (probability distribution functions) in week 2.   The intuitive consequences of these results are:

\begin{itemize}
\item (Non-negative) Probability can never be negative
\item (Total probability) The probability of a sample space must equal 1 
\item (Countable additivity) The probability of observing two (or more) mutually exclusive events is the sum of their individual probabilities
\end{itemize}

However, we need to derive some additional results from these Axioms in order to be able to carry out useful probability calculations.



\subsection{Deriving $P(A^C)$}

The sets $A$ and $A^C$ form a partition of the sample space so that $\Omega = A \cup A^C$.   We therefore know by the second axiom that $P(A \cup A^C) = 1$.   By the third axiom we know that $P(A \cup A^C) = P(A) + P(A^C)$ so we can rearrange this to give:

\begin{align*}
P(A \cup A^C) &= P(A) + P(A^C) \\
P(A \cup A^C) - P(A) &= P(A^C) \\
P(A^C) &= 1 - P(A)
\end{align*}

\subsection{Deriving $P(A) \leq 1$}

If $P(A^C) \geq 0$ and $P(A) + P(A^C) = 1$ this is immediate.

\subsection{Deriving the addition rule}

Consider the set $B$, which we can expand as $B=(B \cap A) \cup (B \cap A^C)$.   This tells us (eventually) that

\begin{displaymath}
P(B) = P(B \cap A) + P(B \cap A^C)
\end{displaymath}
Note that the two terms on the right hand side are disjoint so we can use the third axiom.

Now we want to think about $A \cup B = A \cup (B \cap A^C)$, again the two terms on the right hand side are mutually exclusive/disjoint.   

\begin{align*}
P(A \cup B) &= P(A) + P(B \cap A^C) \\
 &= P(A) + P(B) - P(A \cap B)
 \end{align*}



You can verify this; draw the Venn Diagram.   If we merely added $p[A]$ to $p[B]$ for events with an intersection, we would add the probability corresponding to $p[A \cap B]$ twice, and hence we need to subtract one of these areas.

{\color{green}What does this tell us about the value of $p[A \cap B]$ for for mutually exclusive events?}


\section{Problems}
\begin{enumerate}
\input{Problems/dummy1/Set1_AdditionRuleProof.tex}
\input{Problems/dummy1/Set1_AdditionRule.tex}
\end{enumerate}

\chapter{Counting rules OK}

\chapter{A non-naive definition of probability}

\chapter{Conditional Probability}

\chapter{Discrete Probability Functions}

\chapter{Continuous Probability Functions}

\end{document}
